% !TEX program = xelatex

\documentclass{resume}
\usepackage{graphicx}
\usepackage{tabu}
\usepackage{multirow}
\usepackage{ctex}
\usepackage{color}
\usepackage{progressbar}
\usepackage{hyperref}

\begin{document}

\begin{center} 
\Large\textbf{明廷来个人简历}
\end{center}
\pagenumbering{gobble} % 隐藏页码

{
% 个人信息栏
\Large{
  \begin{tabu}{ c l }
   \multirow{5}{1in}{\includegraphics[width=0.88in]{mingtinglai}} 
   & \scshape{\textbf{明廷来(Ming Tinglai)}} \\ 
    & \email{tlming16@fudan.edu.cn} \\ 
    & \phone{(+86) 18707147981} \\
    & \github[github.com/tlming16]{https://github.com/tlming16} 
  \end{tabu}
}
}

\section{\textcolor{blue}{\faGraduationCap\ 教育背景}}
\datedsubsection{\textbf{复旦大学}}{2016.9 -- 2019.6}
\textit{应用数学,理学硕士} \quad 
\begin{itemize}
  \item 研究方向:最优化理论、数值分析
  \item 主修课程:凸优化、矩阵计算、机器学习数学基础
\end{itemize}

\datedsubsection{\textbf{华中师范大学}}{2012.9 -- 2016.6}
\textit{历史学基地班(学士)},辅修数学
\begin{itemize}
  \item 主修历史学,系统训练逻辑思维与文献分析能力
  \item 辅修数学专业课程,奠定扎实数理基础
\end{itemize}

\section{\textcolor{blue}{\faBriefcase\ 工作经历}}
\datedsubsection{\textbf{墨芯人工智能 · 大模型编译器工程师}}{上海 \quad 2024.12 -- 至今}
\begin{itemize}
  \item 负责墨芯AI芯片的编译器前端开发与性能优化
  \item 主导大模型算子库建设与硬件适配,支撑LLM训练推理加速
\end{itemize}

\datedsubsection{\textbf{安居客 · 高级算法工程师}}{上海 \quad 2020.11 -- 2024.12}
\begin{itemize}
  \item 负责房产多媒体内容(视频/图片)的算法研发与系统建设
  \item 主导图像审核、检索、结构化分析等核心算法落地
\end{itemize}

\datedsubsection{\textbf{旷视科技 · 研究员}}{北京 \quad 2019.7 -- 2020.10}
\begin{itemize}
  \item 参与视频结构化算法研发,实现人物、车辆、行为的智能解析
  \item 优化多目标跟踪与行为识别模型,提升场景理解准确率
\end{itemize}

\section{\textcolor{blue}{\faFolderOpen\ 项目经历}}
\begin{enumerate}
  \item \textbf{墨芯AI芯片算子库开发(2024.12-至今)}
        \begin{itemize}
          \item 基于Triton框架从零构建墨芯二代芯片前端算子库,涵盖MatMul、FlashAttention(MHA/MQA/GQA)、PagedAttention、MLA等核心算子
          \item 设计片上内存动态调度策略,优化数据复用与带宽利用率,支撑百亿参数模型高效推理
          \item 实现YOLOv8系列(检测/分割/姿态)与ResNet ReID模型在墨芯一代芯片的ONNX编译部署,端侧推理速度提升40\%
        \end{itemize}
        
  \item \textbf{房产图像智能审核系统(2020.11-2024.12)}
        \begin{itemize}
          \item 构建房源图片合规审核系统,基于图像分类算法自动识别违规内容,审核效率提升90\%+
          \item 研发经纪人头像认证系统,融合人脸关键点与欧拉角验证,认证准确率达99.2\%
          \item 开发户型图智能检测与聚类系统,采用YOLO+OCR技术实现户型元素识别(准确率96.5\%)与相似度匹配
          \item 搭建跨平台图片一致性校验系统,基于局部特征比对识别篡改与重复图片
          \item 部署视觉-语言多模态大模型(GLM4V-9B、InternVL2-8B),完成房产场景LoRA微调,支持图文问答与内容生成
        \end{itemize}
        
  \item \textbf{技术架构升级与性能优化(2022-2024)}
        \begin{itemize}
          \item 主导服务端从Python2至Python3的全栈迁移,通过Pybind11实现核心模块C++加速,系统性能提升40\%
          \item 为Java服务端封装OpenCV编译库,提供跨语言调用支持
          \item 搭建Faiss向量检索引擎,实现千万级全景图片毫秒级相似检索
        \end{itemize}
\end{enumerate}

\section{\textcolor{blue}{\faTrophy\ 竞赛获奖}}
\begin{itemize}
  \item \textbf{2021年}:58到家派单算法大赛 \quad \textit{一等奖(第1名)}
  \item \textbf{2023年}:信息茧房算法挑战赛 \quad \textit{三等奖(第4名)}
  \item \textbf{2022年}:图像拼接算法大赛 \quad \textit{三等奖(第5名)}
  \item \textbf{2020年}:域外探索算法竞赛 \quad \textit{第十名}
\end{itemize}

\section{\textcolor{blue}{\faCogs\ 专业技能}}
\begin{itemize}
  \item \textbf{编程语言}:精通 Python/C++,熟练使用 Java/Go/Julia/D/Rust,熟悉 STL、LLVM、MLIR
  \item \textbf{系统开发}:熟悉 Linux 环境、GCC工具链、Git 版本控制,掌握多线程编程(OpenMP/TBB/CUDA)
  \item \textbf{算法基础}:扎实的数据结构与算法基础,熟悉计算机视觉、深度学习、数学优化理论
  \item \textbf{工具框架}:熟练使用 PyTorch/TensorFlow、ONNX Runtime、Triton、LaTeX
  \item \textbf{英语能力}:CET-6(516分),具备英文技术文献阅读能力(研究生英语免修)
\end{itemize}

\section{\textcolor{red}{\faHeartO\ 个人兴趣}}
\begin{itemize}
  \item \textbf{编程语言研究}:深入探索 Go、Julia、D、Rust 等现代语言的特性与设计哲学
  \item \textbf{开源项目学习}:研读 OpenCV、TensorFlow、LLVM、Boost 等大型 C++ 项目源码
  \item \textbf{优化理论应用}:关注凸优化、自动微分在最优化与机器学习中的前沿应用
\end{itemize}

\end{document}